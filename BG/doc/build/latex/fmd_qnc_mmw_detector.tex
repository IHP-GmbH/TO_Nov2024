%% Generated by Sphinx.
\def\sphinxdocclass{report}
\documentclass[letterpaper,10pt,english]{sphinxmanual}
\ifdefined\pdfpxdimen
   \let\sphinxpxdimen\pdfpxdimen\else\newdimen\sphinxpxdimen
\fi \sphinxpxdimen=.75bp\relax
\ifdefined\pdfimageresolution
    \pdfimageresolution= \numexpr \dimexpr1in\relax/\sphinxpxdimen\relax
\fi
%% let collapsible pdf bookmarks panel have high depth per default
\PassOptionsToPackage{bookmarksdepth=5}{hyperref}

\PassOptionsToPackage{booktabs}{sphinx}
\PassOptionsToPackage{colorrows}{sphinx}

\PassOptionsToPackage{warn}{textcomp}
\usepackage[utf8]{inputenc}
\ifdefined\DeclareUnicodeCharacter
% support both utf8 and utf8x syntaxes
  \ifdefined\DeclareUnicodeCharacterAsOptional
    \def\sphinxDUC#1{\DeclareUnicodeCharacter{"#1}}
  \else
    \let\sphinxDUC\DeclareUnicodeCharacter
  \fi
  \sphinxDUC{00A0}{\nobreakspace}
  \sphinxDUC{2500}{\sphinxunichar{2500}}
  \sphinxDUC{2502}{\sphinxunichar{2502}}
  \sphinxDUC{2514}{\sphinxunichar{2514}}
  \sphinxDUC{251C}{\sphinxunichar{251C}}
  \sphinxDUC{2572}{\textbackslash}
\fi
\usepackage{cmap}
\usepackage[T1]{fontenc}
\usepackage{amsmath,amssymb,amstext}
\usepackage{babel}



\usepackage{tgtermes}
\usepackage{tgheros}
\renewcommand{\ttdefault}{txtt}



\usepackage[Bjarne]{fncychap}
\usepackage{sphinx}

\fvset{fontsize=auto}
\usepackage{geometry}


% Include hyperref last.
\usepackage{hyperref}
% Fix anchor placement for figures with captions.
\usepackage{hypcap}% it must be loaded after hyperref.
% Set up styles of URL: it should be placed after hyperref.
\urlstyle{same}


\usepackage{sphinxmessages}




\title{FMD\_QNC\_mmW\_detector}
\date{Dec 18, 2024}
\release{0.1}
\author{oliver munz}
\newcommand{\sphinxlogo}{\vbox{}}
\renewcommand{\releasename}{Release}
\makeindex
\begin{document}

\ifdefined\shorthandoff
  \ifnum\catcode`\=\string=\active\shorthandoff{=}\fi
  \ifnum\catcode`\"=\active\shorthandoff{"}\fi
\fi

\pagestyle{empty}
\sphinxmaketitle
\pagestyle{plain}
\sphinxtableofcontents
\pagestyle{normal}
\phantomsection\label{\detokenize{index::doc}}


\begin{sphinxadmonition}{warning}{Warning:}
\sphinxAtStartPar
this chip is mostly done using an unfinished PDK. LVS was done only for parts.
\end{sphinxadmonition}

\noindent{\hspace*{\fill}\sphinxincludegraphics{{CMOS_chip}.png}\hspace*{\fill}}

\sphinxAtStartPar


\sphinxstepscope


\chapter{CMOS only designs}
\label{\detokenize{specification:cmos-only-designs}}\label{\detokenize{specification::doc}}
\sphinxAtStartPar
this is a chip to test the following circuits:
\begin{enumerate}
\sphinxsetlistlabels{\arabic}{enumi}{enumii}{}{.}%
\item {} \begin{description}
\sphinxlineitem{an 3.3V OTA}\begin{itemize}
\item {} 
\sphinxAtStartPar
an input common\sphinxhyphen{}mode\sphinxhyphen{}range from \textless{} VSS to VDD \sphinxhyphen{} 500mV

\item {} 
\sphinxAtStartPar
bias current programmable via resistor

\item {} 
\sphinxAtStartPar
gain about 70dB

\item {} 
\sphinxAtStartPar
offset in \textless{} 5mV

\end{itemize}

\end{description}

\item {} \begin{description}
\sphinxlineitem{a band gap design using this OTA.}\begin{itemize}
\item {} 
\sphinxAtStartPar
option to make an low\sphinxhyphen{}pass

\item {} 
\sphinxAtStartPar
option to adjust the delta temperature coefficient zero

\end{itemize}

\end{description}

\item {} \begin{description}
\sphinxlineitem{a shunt\sphinxhyphen{}regulator for 3.3V}\begin{itemize}
\item {} 
\sphinxAtStartPar
should heat until 50mA

\item {} 
\sphinxAtStartPar
option to disable and measure the shunt current

\end{itemize}

\end{description}

\item {} \begin{description}
\sphinxlineitem{a high dropout 1.8V regulator}\begin{itemize}
\item {} 
\sphinxAtStartPar
should be powered by the 3.3V

\item {} 
\sphinxAtStartPar
low output impedance

\item {} 
\sphinxAtStartPar
high power supply rejection

\end{itemize}

\end{description}

\item {} \begin{description}
\sphinxlineitem{two different mmW detector designs}\begin{itemize}
\item {} 
\sphinxAtStartPar
dipole/diode detector

\item {} 
\sphinxAtStartPar
a 1 diode design as reference of a

\item {} 
\sphinxAtStartPar
10 serial connected diode design

\end{itemize}

\end{description}

\item {} 
\sphinxAtStartPar
iHPs standard IO\sphinxhyphen{}cells for analog signal, VSS, VDD and IOVSS, IOVDD

\item {} 
\sphinxAtStartPar
a schottky diode, for measurements

\item {} 
\sphinxAtStartPar
a fuse/resistor combination to test the possibility of fuse\sphinxhyphen{}trimming

\end{enumerate}


\section{pinout}
\label{\detokenize{specification:pinout}}
\noindent{\hspace*{\fill}\sphinxincludegraphics{{pinout}.eps}\hspace*{\fill}}


\section{signals}
\label{\detokenize{specification:signals}}

\begin{savenotes}\sphinxattablestart
\sphinxthistablewithglobalstyle
\centering
\begin{tabulary}{\linewidth}[t]{TT}
\sphinxtoprule
\sphinxstyletheadfamily 
\sphinxAtStartPar
pin name
&\sphinxstyletheadfamily 
\sphinxAtStartPar
whats it
\\
\sphinxmidrule
\sphinxtableatstartofbodyhook\sphinxstyletheadfamily 
\sphinxAtStartPar
fuse
&
\sphinxAtStartPar
fuse experiment
\\
\sphinxhline\sphinxstyletheadfamily 
\sphinxAtStartPar
det
&
\sphinxAtStartPar
dipole detector
\\
\sphinxhline\sphinxstyletheadfamily 
\sphinxAtStartPar
det10
&
\sphinxAtStartPar
10x serial dipole detectors
\\
\sphinxhline\sphinxstyletheadfamily 
\sphinxAtStartPar
schottky
&
\sphinxAtStartPar
test\sphinxhyphen{}diode
\\
\sphinxhline\sphinxstyletheadfamily 
\sphinxAtStartPar
OTA+
&
\sphinxAtStartPar
non\sphinxhyphen{}inverting OTA input
\\
\sphinxhline\sphinxstyletheadfamily 
\sphinxAtStartPar
OTA\sphinxhyphen{}
&
\sphinxAtStartPar
inverting OTA input
\\
\sphinxhline\sphinxstyletheadfamily 
\sphinxAtStartPar
OTAo
&
\sphinxAtStartPar
OTA output
\\
\sphinxhline\sphinxstyletheadfamily 
\sphinxAtStartPar
shuntGND
&
\sphinxAtStartPar
shunt\sphinxhyphen{}regulators VSS connection. can be used to measure the current
\\
\sphinxhline\sphinxstyletheadfamily 
\sphinxAtStartPar
IOVDD
&
\sphinxAtStartPar
3.3V input and shunt\sphinxhyphen{}cathode if shuntGND is connected to IOVSS
\\
\sphinxhline\sphinxstyletheadfamily 
\sphinxAtStartPar
IOVSS
&
\sphinxAtStartPar
0V \sphinxhyphen{} connect to VSS
\\
\sphinxhline\sphinxstyletheadfamily 
\sphinxAtStartPar
VSS
&
\sphinxAtStartPar
0V \sphinxhyphen{} connect to IOVSS
\\
\sphinxhline\sphinxstyletheadfamily 
\sphinxAtStartPar
VDD
&
\sphinxAtStartPar
1.8V regulator output
\\
\sphinxhline\sphinxstyletheadfamily 
\sphinxAtStartPar
bias in
&
\sphinxAtStartPar
resistor from IOVDD sets the bias\sphinxhyphen{}currents of all OTAs
\\
\sphinxhline\sphinxstyletheadfamily 
\sphinxAtStartPar
ref out
&
\sphinxAtStartPar
band\sphinxhyphen{}gap voltage \sphinxhyphen{} high impedance output and regulator ref input. low\sphinxhyphen{}pass possible.
\\
\sphinxhline\sphinxstyletheadfamily 
\sphinxAtStartPar
ref cur
&
\sphinxAtStartPar
reference current source from IOVDD
\\
\sphinxhline\sphinxstyletheadfamily 
\sphinxAtStartPar
ref adj
&
\sphinxAtStartPar
resistor to IOVDD or IOVSS to trim the zero of the temperature coefficient
\\
\sphinxbottomrule
\end{tabulary}
\sphinxtableafterendhook\par
\sphinxattableend\end{savenotes}

\sphinxstepscope


\chapter{high voltage OTA}
\label{\detokenize{ota:high-voltage-ota}}\label{\detokenize{ota::doc}}
\sphinxAtStartPar
1.1. the OTA
the circuit should be used with iHPs PNP\sphinxhyphen{}device pnpMPA, so it should be working from VSS. this called for an PMOS input. the design is a simplified version of
\sphinxcode{\sphinxupquote{The Recycling Folded Cascode: A General
Enhancement of the Folded Cascode Amplifier}}

\sphinxAtStartPar
because of the used CMOS\sphinxhyphen{}process the PMOS transistors could have an isolated bulk, without special effort. the simulations showed that the isolated versions had a bigger gain, but a smaller common\sphinxhyphen{}mode\sphinxhyphen{}range, and i preferred the later.
the bias\sphinxhyphen{}current is programmable an so also the band\sphinxhyphen{}width. the bias\sphinxhyphen{}voltages should allow wide\sphinxhyphen{}swing output voltages.
in the space the circuit uses are 3 MiM\sphinxhyphen{}capacitors placed. they are intended to use for frequency\sphinxhyphen{}compensation or as power\sphinxhyphen{}rail decoupling.


\section{OTA}
\label{\detokenize{ota:ota}}
\noindent{\hspace*{\fill}\sphinxincludegraphics{{OTA3C}.eps}\hspace*{\fill}}


\section{bias generator}
\label{\detokenize{ota:bias-generator}}
\noindent{\hspace*{\fill}\sphinxincludegraphics{{OTA33_BiAS}.eps}\hspace*{\fill}}


\section{layout}
\label{\detokenize{ota:layout}}
\noindent{\hspace*{\fill}\sphinxincludegraphics{{OTA_layout}.png}\hspace*{\fill}}

\sphinxAtStartPar



\section{simulations}
\label{\detokenize{ota:simulations}}
\sphinxAtStartPar
using different bias\sphinxhyphen{}currents of 1, 3 and 10µA a few simulations are printed into a PDF that allow to see the gain, common\sphinxhyphen{}mode\sphinxhyphen{}range, bandwidth and a slew\sphinxhyphen{}rate:

\sphinxAtStartPar
\sphinxcode{\sphinxupquote{PDF with Xyce simulations .ac .dc .trans}}

\sphinxAtStartPar
the schematics of this simulations is \sphinxcode{\sphinxupquote{Xschem document}}


\section{ETHZ feedback}
\label{\detokenize{ota:ethz-feedback}}
\sphinxAtStartPar
its a bit stupid to design OTAs that fit in a square, if there is no such space\sphinxhyphen{}requirement. the layout should be changed to minimize the conductor\sphinxhyphen{}length of the signals between the differential\sphinxhyphen{}stage and the current\sphinxhyphen{}mirrors.

\sphinxstepscope


\chapter{band gap reference}
\label{\detokenize{reference:band-gap-reference}}\label{\detokenize{reference::doc}}
\sphinxAtStartPar
the shunt\sphinxhyphen{}regulator has to limit the supply\sphinxhyphen{}voltage to 3.3V. it should be usable to 50mA.


\section{schematic}
\label{\detokenize{reference:schematic}}
\noindent{\hspace*{\fill}\sphinxincludegraphics{{bg}.eps}\hspace*{\fill}}


\section{layout}
\label{\detokenize{reference:layout}}
\noindent{\hspace*{\fill}\sphinxincludegraphics{{bg_layout}.png}\hspace*{\fill}}


\section{simulation}
\label{\detokenize{reference:simulation}}
\sphinxAtStartPar
\sphinxcode{\sphinxupquote{PDF with Xyce simulation}}


\section{LVS}
\label{\detokenize{reference:lvs}}
\sphinxAtStartPar
LVS wasn’t working. and there are differences between layout and schematics

\sphinxstepscope


\chapter{voltage\sphinxhyphen{}regulators}
\label{\detokenize{regulators:voltage-regulators}}\label{\detokenize{regulators::doc}}

\section{shunt\sphinxhyphen{}regulator}
\label{\detokenize{regulators:shunt-regulator}}
\sphinxAtStartPar
the shunt\sphinxhyphen{}regulator has to limit the supply\sphinxhyphen{}voltage to 3.3V. it should be usable to 50mA. the regulating current flows thru M11. and to allow this the pin SHUNT\_GND needs to be connected to VSS. optional over an current\sphinxhyphen{}meter.

\sphinxAtStartPar
in the layout the left/top OTA is used in this circuit.


\section{serial\sphinxhyphen{}regulator}
\label{\detokenize{regulators:serial-regulator}}
\sphinxAtStartPar
the serial regulator should produce 1.8V for low\sphinxhyphen{}voltage MOSFETs. the OTA responsible for this regulator is left/bottom in the layout.


\section{schematic}
\label{\detokenize{regulators:schematic}}
\noindent{\hspace*{\fill}\sphinxincludegraphics{{regs}.eps}\hspace*{\fill}}


\section{layout}
\label{\detokenize{regulators:layout}}
\noindent{\hspace*{\fill}\sphinxincludegraphics{{regs_layout}.png}\hspace*{\fill}}


\section{simulations}
\label{\detokenize{regulators:simulations}}
\sphinxAtStartPar
\sphinxcode{\sphinxupquote{PDF with Xyce simulation}}


\section{LVS}
\label{\detokenize{regulators:lvs}}
\sphinxAtStartPar
LVS wasn’t working

\sphinxstepscope


\chapter{iHPs IO\sphinxhyphen{}CELLs}
\label{\detokenize{io:ihps-io-cells}}\label{\detokenize{io::doc}}
\sphinxAtStartPar
for ESD protection the analog\sphinxhyphen{}IO\sphinxhyphen{}cells where used. the cells support two voltages. a 3.3V IO and a 1.8V core\sphinxhyphen{}voltage.

\sphinxAtStartPar
my problem using the cells was, that the openPDK LVS scripts didn’t could verify the .GDS with the provided .CDL. so that simply hope, they are correct used.

\sphinxAtStartPar
one problem i had, using this cells was the space they demand. i was trying to make a minimal chip of about 0.5x1mm. but the filler.py\sphinxhyphen{}script wasn’t able to produce enough GatPoly\sphinxhyphen{}fill to pass the maximum\sphinxhyphen{}DRC script.

\sphinxAtStartPar
this IO infrastructure needs at least a 1x1mm chip (according my experience).

\sphinxstepscope


\chapter{mmW diode detectors}
\label{\detokenize{detectors:mmw-diode-detectors}}\label{\detokenize{detectors::doc}}
\sphinxAtStartPar
there are two diode\sphinxhyphen{}detector designs to measure if serial circuits of detector diodes will increase the sensitivity of the sensor.


\section{layout}
\label{\detokenize{detectors:layout}}
\noindent{\hspace*{\fill}\sphinxincludegraphics{{detectors}.png}\hspace*{\fill}}

\sphinxAtStartPar


\sphinxstepscope


\chapter{experiments}
\label{\detokenize{tests:experiments}}\label{\detokenize{tests::doc}}

\section{test schottky diode}
\label{\detokenize{tests:test-schottky-diode}}
\sphinxAtStartPar
to measure the schottky\sphinxhyphen{}diode used in the mmW\sphinxhyphen{}detectors, a device is connected to bond\sphinxhyphen{}pads outside the power\sphinxhyphen{}ring.


\section{fuse}
\label{\detokenize{tests:fuse}}
\sphinxAtStartPar
to test the possibility of using fuses to trim band\sphinxhyphen{}gap\sphinxhyphen{}references and the like a test\sphinxhyphen{}structure with a resistor, parallel to a fuse, is connected to two pads outside the power\sphinxhyphen{}ring.

\sphinxstepscope


\chapter{design data and design process description}
\label{\detokenize{designdata:design-data-and-design-process-description}}\label{\detokenize{designdata::doc}}

\section{iHP 130nm BiCMOS process sg13g2}
\label{\detokenize{designdata:ihp-130nm-bicmos-process-sg13g2}}
\sphinxAtStartPar
the process is useable via iHPs openPDK:

\sphinxAtStartPar
source
\sphinxurl{https://github.com/IHP-GmbH/IHP-Open-PDK}

\sphinxAtStartPar
documentation
\sphinxurl{https://ihp-open-pdk-docs.readthedocs.io/en/latest/index.html}

\sphinxAtStartPar
open\sphinxhyphen{}source runs:
\sphinxurl{https://www.ihp-microelectronics.com/services/research-and-prototyping-service/fast-design-enablement/open-source-pdk}


\section{Xschem schematics:}
\label{\detokenize{designdata:xschem-schematics}}

\begin{savenotes}\sphinxattablestart
\sphinxthistablewithglobalstyle
\centering
\begin{tabulary}{\linewidth}[t]{TT}
\sphinxtoprule
\sphinxstyletheadfamily 
\sphinxAtStartPar
folder
&\sphinxstyletheadfamily 
\sphinxAtStartPar
for
\\
\sphinxmidrule
\sphinxtableatstartofbodyhook\sphinxstyletheadfamily 
\sphinxAtStartPar
design\_data/xschem/cdl
&
\sphinxAtStartPar
symbols for Xschem schematics
\\
\sphinxhline\sphinxstyletheadfamily 
\sphinxAtStartPar
design\_data/xschem/OTA
&
\sphinxAtStartPar
design and simulation of the OTA
\\
\sphinxhline\sphinxstyletheadfamily 
\sphinxAtStartPar
design\_data/xschem/simulations
&
\sphinxAtStartPar
simulations of the voltage regulators
\\
\sphinxhline\sphinxstyletheadfamily 
\sphinxAtStartPar
design\_data/xschem/symbol
&
\sphinxAtStartPar
symbols for Xschem schematics
\\
\sphinxbottomrule
\end{tabulary}
\sphinxtableafterendhook\par
\sphinxattableend\end{savenotes}


\section{KLayout .GDS:}
\label{\detokenize{designdata:klayout-gds}}

\begin{savenotes}\sphinxattablestart
\sphinxthistablewithglobalstyle
\centering
\begin{tabulary}{\linewidth}[t]{TT}
\sphinxtoprule
\sphinxstyletheadfamily 
\sphinxAtStartPar
file
&\sphinxstyletheadfamily 
\sphinxAtStartPar
for
\\
\sphinxmidrule
\sphinxtableatstartofbodyhook\sphinxstyletheadfamily 
\sphinxAtStartPar
design\_data/gds/FMD\_QNC\_mmW\_detector.gds.gz
&
\sphinxAtStartPar
layout of the chip
\\
\sphinxbottomrule
\end{tabulary}
\sphinxtableafterendhook\par
\sphinxattableend\end{savenotes}

\sphinxstepscope


\chapter{validation sometime in the farfar future}
\label{\detokenize{validation:validation-sometime-in-the-farfar-future}}\label{\detokenize{validation::doc}}
\sphinxAtStartPar
i cant wait…

\begin{sphinxcontents}
\sphinxstylecontentstitle{Contents}
\begin{itemize}
\item {} 
\sphinxAtStartPar
\phantomsection\label{\detokenize{index:id1}}{\hyperref[\detokenize{index:cmos-only-bicmos-chip}]{\sphinxcrossref{CMOS only BiCMOS chip ;)}}}
\begin{itemize}
\item {} 
\sphinxAtStartPar
\phantomsection\label{\detokenize{index:id2}}{\hyperref[\detokenize{index:background}]{\sphinxcrossref{background}}}

\item {} 
\sphinxAtStartPar
\phantomsection\label{\detokenize{index:id3}}{\hyperref[\detokenize{index:the-sg13g2-process}]{\sphinxcrossref{the sg13g2 process}}}

\end{itemize}

\end{itemize}
\end{sphinxcontents}


\chapter{background}
\label{\detokenize{index:background}}
\sphinxAtStartPar
to make first steps using iHPs openPDK, Xschem and KLayout designing a OTA was obvious. this part is DRC and LVS clean, verified also using commercial tools.

\sphinxAtStartPar
but that’s also the only clean part on this chip from my perspective. its highly likely that the IO cells are perfect, but i couldn’t verify this my self. the opensource LVS didn’t was ready, and i hadn’t the time to verify any other part then the OTA via LVS.


\chapter{the sg13g2 process}
\label{\detokenize{index:the-sg13g2-process}}
\sphinxAtStartPar
there are two kinds of GateOxyde: thick for high\sphinxhyphen{}voltage like 3.3V circuits and standard for 1.2..1.8V circuits. there are three different SiGe BjTs, but all are NPNs. while the Si\sphinxhyphen{}parasitic transistors are only PNP.



\renewcommand{\indexname}{Index}
\printindex
\end{document}